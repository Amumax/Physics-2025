
\documentclass[12pt,a4paper]{article}
\usepackage[utf8]{inputenc}
\usepackage[russian]{babel}
\usepackage{amsmath,amssymb}
\usepackage{graphicx}
\usepackage{hyperref}
\usepackage{geometry}
\geometry{margin=2.5cm}

\title{Молекулярная динамика идеального и неидеального газа \\
       \large{Отчёт по задачам M5, 5A, 5Б, 5В}}
\author{Найденов Максим \\ Осипчук Лидия}
\date{\today}

\begin{document}
\maketitle

\tableofcontents
\newpage

\section{Введение}
В работе проведено молекулярно-динамическое моделирование идеального и неидеального газа в двумерном сосуде с различными физическими условиями. 
Цель --- продемонстрировать выполнение кинетических законов (уравнение состояния, распределение Максвелла--Больцмана), а также исследовать адиабатические, неидеальные и гравитационные эффекты.

\section{Теоретические основы}
\input{theory.tex}

\section{Методы}
\input{methods.tex}

\section{Результаты}
\subsection{Задача M5: Идеальный газ в сосуде}
% Вставить рисунки: скриншот анимации, графики P(t), T(t), распределение скоростей
\subsection{Задача 5A: Адиабатическое сжатие/расширение}
\subsection{Задача 5Б: Неидеальный газ с потенциалом Леннарда--Джонса}
\subsection{Задача 5В: Газовый столб в поле тяжести}

\section{Заключение}
Симуляции подтвердили основные положения статистической физики: ...

\end{document}
