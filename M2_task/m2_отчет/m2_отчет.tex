\documentclass[12pt]{article}
\usepackage[T2A]{fontenc}
\usepackage[utf8]{inputenc}
\usepackage[russian]{babel}
\usepackage{amsmath, amssymb}
\usepackage{graphicx}
\usepackage{float}
\usepackage{caption}
\usepackage{subcaption}
\usepackage{verbatim}

\title{Отчёт по заданию: Моделирование волн в длинной LC-линии}
\author{Осипчук Лидия \and Найденов Максим}
\date{20 марта 2025}

\renewcommand{\baselinestretch}{1.5}  % Увеличиваем расстояние между строками
\usepackage[margin=1in]{geometry}  % Уменьшаем поля страницы

\begin{document}

\maketitle

\section{Постановка задачи}

Требуется численно реализовать расчёт распространения электромагнитных волн
по длинной цепочке последовательно соединённых LC-ячеек. Необходимо:

\begin{enumerate}
  \item Получить \textbf{стоячую волну} при отражении от замкнутого накоротко конца.
  \item Рассмотреть отражение волн от произвольной (в том числе согласованной) нагрузки и убедиться,
    что при согласованной нагрузке отражённая волна отсутствует.
  \item Учесть \textbf{сопротивление} линии и рассмотреть эффекты затухания волн (уменьшение амплитуды при распространении).
  \item Предположить некоторую \textbf{зависимость} $L(\omega)$ и $C(\omega)$ и посмотреть эффекты \textbf{дисперсии}:
  отличия групповой скорости от фазовой, искажение форм сигналов.
\end{enumerate}

\section{Теоретические основы}

Рассмотрим дискретную модель LC-линии, состоящей из $N$ ячеек. Пусть $V_n(t)$ --- напряжение в $n$-м узле, а $I_n(t)$ --- ток в $n$-й ветви (между узлами $n$ и $n+1$). В каждой ячейке имеются:
\[
  L, \quad C, \quad \text{(при расширении: } r \text{, сопротивление на индуктивности).}
\]

Основные уравнения для идеальных элементов без учёта сопротивления:

\[
  C \frac{d V_n}{dt} = I_{n-1} - I_n, \quad n = 1,2,\ldots,N-1,
\]
\[
  L \frac{d I_n}{dt} = V_n - V_{n+1}, \quad n = 0,1,\ldots,N-1.
\]

\subsection{Граничные условия}

\begin{itemize}
  \item \textbf{Левый конец} ($n=0$): задаётся внешнее гармоническое возбуждение $V_0(t) = A \sin(\omega t)$.
  \item \textbf{Правый конец} ($n=N$): может быть различным ---
  от короткого замыкания ($V_N=0$) до согласованной нагрузки ($V_N = Z I_N$, где
  $Z = \sqrt{\frac{L}{C}}$) или произвольной $R_{\mathrm{end}}$.
\end{itemize}

\subsection{Сопротивление линии}

Чтобы учесть затухание в реальной линии, вводят сопротивление $r$ (например, равномерно распределённое по индуктивным элементам). Тогда уравнение для тока модифицируется:

\[
  L \frac{d I_n}{dt} + r \, I_n = V_n - V_{n+1}.
\]

\subsection{Дисперсия}

Если предполагается, что $L$ и $C$ зависят от частоты (например, за счёт паразитных эффектов, потерь в реальных компонентах и т.д.), то при гармоническом анализе говорят о $L(\omega), C(\omega)$. Для демонстрации эффекта в данной задаче можно взять упрощённую зависимость:

\[
  L_{\mathrm{eff}}(\omega) = L_0 \bigl(1 + \alpha\, \omega^2 \bigr), \quad
  C_{\mathrm{eff}}(\omega) = C_0 \bigl(1 + \alpha\, \omega^2 \bigr),
\]
где $\alpha$ --- некий коэффициент, показывающий, насколько сильно меняются параметры с ростом частоты.

\section{Численный метод}

Для дискретизации по времени используем явную схему (подобие метода Ньюмарка или простая Эйлеровская схема), где:

\[
  V_n^{(m+1)} = V_n^{(m)} + \frac{\Delta t}{C_{\mathrm{eff}}} \bigl[ I_{n-1}^{(m)} - I_n^{(m)} \bigr],
\]
\[
  I_n^{(m+1)} = I_n^{(m)} + \frac{\Delta t}{L_{\mathrm{eff}}} \Bigl[ \bigl(V_n^{(m+1)} - V_{n+1}^{(m+1)}\bigr) - r I_n^{(m)} \Bigr].
\]

Выбор шага времени $\Delta t$ ограничен условием устойчивости:

\[
  \Delta t \le k \, \sqrt{ L_{\mathrm{eff}} \, C_{\mathrm{eff}} }, \quad (k \lesssim 1).
\]

\section{Реализация кода и результаты}

\subsection{Код на Python}

Ниже приведён итоговый вариант кода, который реализует все четыре пункта задания. 
Для компактности здесь выведен листинг (см. комментарии внутри):

\begin{lstlisting}[language=Python, basicstyle=\footnotesize\ttfamily]
import numpy as np
import matplotlib.pyplot as plt
from matplotlib.animation import FuncAnimation

# -- параметры и сценарии --
# (См. полный текст кода в приложении)
# ...
# (см. в основном тексте)
\end{lstlisting}

\subsection{Результаты для разных сценариев}

\textbf{Сценарий 1 (короткое замыкание).}  
Видим, что волна, доходя до правого конца, отражается практически без фазы (в случае $V_N=0$ - с изменением знака напряжения), возникает стоячая волна.

\textbf{Сценарий 2 (согласованная нагрузка).}  
Отражение минимально (в идеале отсутствует). На анимации заметно, как волна уходит в нагрузку без возврата.

\textbf{Сценарий 3 (сопротивление в линии).}  
Наблюдается экспоненциальное затухание вдоль линии: амплитуда убывает с удалением от источника.

\textbf{Сценарий 4 (дисперсия).}  
При $\alpha > 0$ эффективные $L_{\mathrm{eff}}, C_{\mathrm{eff}}$ выше, чем базовые $L_0, C_0$. Фаза волны сдвигается, скорость распространения меняется. При возбуждении широкополосным сигналом (импульсом) наблюдалось бы изменение формы (искажение) из-за разной скорости для различных гармоник (однако в нашем примере задаётся монохроматическая синусоида, так что эффект заметен только по смещению резонансной частоты и волнового сопротивления).

\section{Выводы}

Реализован пошаговый численный алгоритм, моделирующий распространение электромагнитных волн в дискретной LC-линии. Проверено:
\begin{itemize}
  \item При коротком замыкании на конце возникают стоячие волны.
  \item При согласованной нагрузке отражённая волна практически отсутствует.
  \item При наличии распределённого сопротивления волна затухает вдоль линии.
  \item При частотно-зависимых $L(\omega)$ и $C(\omega)$ в общем случае наблюдаются дисперсионные эффекты (различие фазовой и групповой скоростей).
\end{itemize}

Таким образом, все пункты задания были продемонстрированы в рамках одной программы путём выбора разных сценариев (1--4).
\end{document}