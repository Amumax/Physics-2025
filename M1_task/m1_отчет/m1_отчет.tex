\documentclass[12pt]{article}
\usepackage[T2A]{fontenc}
\usepackage[utf8]{inputenc}
\usepackage[russian]{babel}
\usepackage{amsmath, amssymb}
\usepackage{graphicx}
\usepackage{float}
\usepackage{caption}
\usepackage{subcaption}
\usepackage{verbatim}

\title{Отчёт о моделировании работы радиопередатчика-радиоприёмника с амплитудной модуляцией}
\author{Осипчук Лидия \and Найденов Максим}
\date{27 февраля 2025}

\renewcommand{\baselinestretch}{1.5}  % Увеличиваем расстояние между строками
\usepackage[margin=1in]{geometry}  % Уменьшаем поля страницы

\begin{document}

\maketitle

\section{Введение}
Данный отчёт посвящён моделированию работы системы радиопередачи и приёма, реализованной на языке Python (код \texttt{m1.py}). Эксперимент основан на принципе амплитудной модуляции (AM) и включает генерацию информационного и несущего сигналов, их комбинирование, цифровой анализ с помощью быстрого преобразования Фурье (БПФ) и последующую обработку для выделения полезного сигнала. В отчёте подробно описаны физические законы, лежащие в основе используемых формул и алгоритмов.

\section{Теоретическая модель и физические основы}
\subsection{Амплитудная модуляция (AM)}
Принцип амплитудной модуляции заключается в изменении амплитуды высокочастотного сигнала (несущей) под воздействием информационного сигнала. Если обозначить информационный сигнал как
\[
m(t) = A_m \cos(2\pi f_m t),
\]
а несущую как
\[
c(t) = A_c \cos(2\pi f_c t),
\]
то модулированный сигнал записывается по формуле:
\[
s(t) = \left[1 + k\, m(t)\right] c(t),
\]
где \(k\) --- коэффициент модуляции, определяющий степень изменения амплитуды несущей под воздействием \(m(t)\). Физическая суть этого процесса заключается в том, что изменяемая амплитуда несущей передаёт информацию, записанную в виде колебаний \(m(t)\).

\subsubsection{Раскрытие произведения и спектральный анализ}
Раскроем выражение для \(s(t)\) с использованием тригонометрических тождеств:
\[
s(t) = A_c \cos(2\pi f_c t) + k\,A_m A_c \cos(2\pi f_m t) \cos(2\pi f_c t).
\]
Применяя формулу для произведения косинусов:
\[
\cos \alpha \cos \beta = \frac{1}{2}\left[\cos(\alpha+\beta) + \cos(\alpha-\beta)\right],
\]
получим:
\[
s(t) = A_c \cos(2\pi f_c t) + \frac{k\,A_m A_c}{2} \left[\cos 2\pi(f_c+f_m)t + \cos 2\pi(f_c-f_m)t\right].
\]
Таким образом, спектр модулированного сигнала содержит:
\begin{itemize}
    \item Центральную (несущую) частоту \(f_c\),
    \item Верхнюю боковую полосу \(f_c+f_m\),
    \item Нижнюю боковую полосу \(f_c-f_m\).
\end{itemize}
Эти боковые полосы являются прямым результатом амплитудной модуляции и являются носителями информационной составляющей.

\subsection{Физика работы RC-фильтра}
В системе приёма для выделения низкочастотного информационного сигнала применяется RC-фильтр. Рассмотрим классическую схему первого порядка, состоящую из резистора \(R\) и конденсатора \(C\). Передаточная функция такого фильтра имеет вид:
\[
H(j\omega) = \frac{1}{1+j\omega RC},
\]
а модуль функции передачи:
\[
|H(j\omega)| = \frac{1}{\sqrt{1+(\omega RC)^2}}.
\]
Данная зависимость показывает, что для низких частот (\(\omega \ll 1/RC\)) амплитуда сигнала практически не ослабляется, тогда как для высоких частот (\(\omega \gg 1/RC\)) сигнал значительно подавляется. Это позволяет эффективно удалить высокочастотные компоненты, оставляя только полезный низкочастотный сигнал.

\subsection{Быстрое преобразование Фурье (БПФ)}
БПФ является алгоритмом для вычисления дискретного преобразования Фурье (ДПФ) и используется для анализа спектрального состава сигналов. Согласно теореме Фурье, любой периодический сигнал можно представить в виде суммы синусоидальных функций. Применяя ДПФ к модулированному сигналу \(s(t)\), можно выделить отдельные частотные компоненты, в том числе несущую и боковые полосы, что даёт возможность оценить качество модуляции и демодуляции.

\section{Реализация эксперимента (код \texttt{m1.py})}
В эксперименте, реализованном в файле \texttt{m1.py}, выполняются следующие этапы:

\subsection{Параметры моделирования}
Определяются основные параметры:
\begin{itemize}
    \item Временной интервал и шаг дискретизации (\(\Delta t\)) для создания временной оси, что необходимо для представления сигнала в дискретном виде.
    \item Частоты: информационный сигнал \(f_m\) и несущая \(f_c\) с условием \(f_c \gg f_m\) для обеспечения разнесённости спектральных компонентов.
    \item Амплитуды \(A_m\) и \(A_c\), а также коэффициент модуляции \(k\), которые задают уровни сигнала и степень их взаимодействия.
\end{itemize}

\subsection{Генерация сигналов и их модуляция}
На основе заданных параметров генерируются:
\begin{itemize}
    \item Информационный сигнал: 
    \[
    m(t) = A_m \cos(2\pi f_m t)
    \]
    \item Несущая: 
    \[
    c(t) = A_c \cos(2\pi f_c t)
    \]
    \item Модулированный сигнал по формуле:
    \[
    s(t) = \left[1 + k\, m(t)\right] c(t)
    \]
\end{itemize}
Эта модель точно следует принципам амплитудной модуляции, описанным выше, и позволяет наблюдать образование боковых полос в спектре сигнала.

\subsection{Цифровая обработка и анализ}
Для анализа и последующей обработки сигнала в эксперименте выполняются следующие операции:
\begin{enumerate}
    \item \textbf{Быстрое преобразование Фурье (БПФ).} Применение алгоритма БПФ позволяет вычислить спектральное распределение сигнала \(s(t)\). Это ключевой этап, позволяющий подтвердить теоретический спектральный состав, включающий центральную частоту и боковые полосы.
    \item \textbf{Применение RC-фильтра.} Для выделения низкочастотной (информационной) составляющей используется цифровая имитация RC-фильтра. Как показано выше, фильтр ослабляет высокочастотные компоненты, что позволяет демодулировать сигнал и получить исходное сообщение.
    \item \textbf{Обратное преобразование Фурье.} После фильтрации выполняется обратное преобразование Фурье для восстановления временной формы сигнала, что демонстрирует эффективность процедуры демодуляции.
    \item \textbf{Нормализация сигнала.} Завершающий этап обработки включает нормализацию восстановленного сигнала, что достигается делением всех отсчётов на максимальное значение амплитуды. Это необходимо для корректного сравнения и визуализации результатов.
\end{enumerate}

\subsection{Визуализация результатов}
Код \texttt{m1.py} использует библиотеку \texttt{matplotlib} для построения графиков, демонстрирующих:
\begin{itemize}
    \item Временные ряды информационного сигнала, несущей и модулированного сигнала.
    \item Спектральные характеристики сигнала после применения БПФ, где явно видны компоненты, соответствующие несущей и боковым полосам.
\end{itemize}

\begin{figure}[H]
\centering
\includegraphics[width=\linewidth]{results.png}
\caption{Графики результатов моделирования}
\label{fig:results}
\end{figure}

\section{Результаты моделирования}
На основе эксперимента получены следующие наблюдения:
\begin{itemize}
    \item Модулированный сигнал корректно формируется согласно теоретической модели, что подтверждается наличием характерных спектральных компонентов: центральной несущей и боковых полос.
    \item Применение RC-фильтра позволяет эффективно выделить низкочастотную информационную составляющую, устраняя высокочастотный шум.
    \item Нормализация сигнала обеспечивает сопоставимость амплитуд при последующем анализе и визуализации.
\end{itemize}
Графики, построенные в ходе эксперимента, демонстрируют временную и спектральную динамику сигналов, что соответствует ожидаемым результатам теоретического описания.

\section{Обсуждение}
Эксперимент подтверждает применимость амплитудной модуляции для передачи информации. Физические принципы, лежащие в основе модуляции, основаны на линейном наложении информационного сигнала на высокочастотную несущую. Применение тригонометрических преобразований позволяет получить спектральное представление, где боковые полосы являются носителями информации. Использование RC-фильтра основывается на законе делителя напряжения в цепи с резистором и конденсатором, что позволяет ослабить высокочастотные компоненты и выделить полезный сигнал. Алгоритм БПФ является эффективным методом для дискретного анализа спектра, что даёт возможность оценить распределение энергии по частотам и подтвердить теоретические предположения.

\section{Заключение}
В рамках данного эксперимента была смоделирована работа системы радиопередачи и приёма с использованием амплитудной модуляции. Реализация эксперимента в Python (файл \texttt{m1.py}) включает генерацию информационного и несущего сигналов, их комбинирование, цифровой анализ посредством БПФ, применение RC-фильтра для демодуляции и нормализацию сигнала. Подробный анализ использованных физических законов и формул показывает, что модель полностью соответствует теоретическим представлениям о процессах амплитудной модуляции, фильтрации и спектрального анализа, что позволяет использовать данную схему для дальнейших исследований и оптимизации систем передачи информации.

\end{document}
