\documentclass[a4paper,12pt]{article}
\usepackage[T1]{fontenc}
\usepackage[utf8]{inputenc}
\usepackage[english,russian]{babel}
\usepackage{amsmath,amssymb,amsfonts}
\usepackage{geometry}
\usepackage{graphicx}
\usepackage{hyperref}
\usepackage{listings} % Для вставки фрагментов кода, если нужно
\geometry{top=2cm, left=2cm, right=2cm, bottom=2cm}

\title{Отчёт по задаче: Моделирование линий, плоскостей и оптических приборов \\(интерпретация через трёхмерные прямые)}
\author{Автор: Иванов И.И.}
\date{\today}

\begin{document}

\maketitle

\section{Постановка задачи}

Требуется изучить следующие аспекты:
\begin{enumerate}
  \item \textbf{Задание прямой} в трёхмерном пространстве в параметрической форме:
  \[
    \mathbf{v}(t) = \mathbf{P} + t\, \mathbf{d},
  \]
  где $\mathbf{P}=(x_0,y_0,z_0)$ --- точка на прямой, $\mathbf{d}=(n,m,p)$ --- направляющий вектор.
  \item \textbf{Пересечение двух прямых} в пространстве --- решение системы
  \[
    \begin{cases}
      x_{01} + t\,n_1 = x_{02} + s\,n_2,\\
      y_{01} + t\,m_1 = y_{02} + s\,m_2,\\
      z_{01} + t\,p_1 = z_{02} + s\,p_2.
    \end{cases}
  \]
  Если решение существует (и прямые не параллельны/не скрещены), получаем точку пересечения $\mathbf{r}^{\star}$.
  \item \textbf{Пересечение прямой с плоскостью}, заданной уравнением
  \[
    A\,x + B\,y + C\,z + D = 0.
  \]
  Подстановка параметрического уравнения прямой даёт значение $t$.
  \item \textbf{Применение} полученных формул к простым оптическим системам (тонкая линза, микроскоп, телескоп) --- путём «трассировки лучей» в приближении 2D/3D.
\end{enumerate}

\section{Теоретические основы}

\subsection{Параметрическая форма прямой}

В трёхмерном пространстве $\mathbb{R}^3$ \textbf{прямая} может быть задана векторно:
\[
  \mathbf{r}(t) = \mathbf{P} + t\, \mathbf{d}, 
\]
где $t\in \mathbb{R}$, $\mathbf{P}=(x_0,y_0,z_0)$, $\mathbf{d}=(d_x,d_y,d_z)$.

Скалярно (по координатам) это
\[
  \begin{cases}
    x(t) = x_0 + t\,d_x, \\
    y(t) = y_0 + t\,d_y, \\
    z(t) = z_0 + t\,d_z.
  \end{cases}
\]

\subsection{Пересечение прямых}

Рассмотрим две прямые:
\[
  \mathbf{r}_1(t) = \mathbf{P}_1 + t\,\mathbf{d}_1, 
  \quad
  \mathbf{r}_2(s) = \mathbf{P}_2 + s\,\mathbf{d}_2.
\]
Пересечение означает, что существует $(t^{\star}, s^{\star})$ такое, что 
\(
  \mathbf{r}_1(t^{\star}) = \mathbf{r}_2(s^{\star}).
\)
В координатном виде:
\[
  \begin{cases}
  x_{01} + t\,d_{1x} = x_{02} + s\,d_{2x},\\
  y_{01} + t\,d_{1y} = y_{02} + s\,d_{2y},\\
  z_{01} + t\,d_{1z} = z_{02} + s\,d_{2z}.
  \end{cases}
\]
Решая эту систему относительно $(t,s)$, находим при наличии решения точку пересечения.

\subsection{Пересечение прямой с плоскостью}

Пусть плоскость задана уравнением
\[
  A\,x + B\,y + C\,z + D = 0.
\]
Подставляя параметры прямой $\mathbf{r}(t) = \mathbf{p} + t\,\mathbf{d}$, получаем скалярное уравнение:
\[
  A(x_0 + t\,d_x) + B(y_0 + t\,d_y) + C(z_0 + t\,d_z) + D = 0,
\]
\[
  t\,(A\,d_x + B\,d_y + C\,d_z) = - \bigl(A\,x_0 + B\,y_0 + C\,z_0 + D\bigr).
\]
Если знаменатель $A\,d_x + B\,d_y + C\,d_z\neq 0$, то
\[
  t^{\star} = -\frac{A\,x_0 + B\,y_0 + C\,z_0 + D}{A\,d_x + B\,d_y + C\,d_z}.
\]
Тогда точка пересечения: 
\(
  \mathbf{r}(t^{\star})=\mathbf{p}+ t^{\star}\mathbf{d}.
\)

\section{Численная реализация и ход лучей в модели <<тонкой линзы>>}

В прилагаемом \texttt{Jupyter Notebook} (см. листинг или файл) мы реализовали классы:
\begin{itemize}
  \item \textbf{\texttt{Line}} --- хранит информацию о прямой (\texttt{offset}, \texttt{multiply}) и методы:
    \begin{itemize}
      \item \texttt{get\_dot(t)} --- получение координат точки при параметре $t$;
      \item \texttt{intersection(other)} --- решение системы для пересечения с другой прямой.
    \end{itemize}
  \item \textbf{\texttt{Plane}} --- задаёт плоскость с параметрами $(A,B,C,D)$ и метод \texttt{intersection\_with\_line(line)}, возвращающий точку пересечения.
\end{itemize}

Для **оптической** задачи (тонкая линза, микроскоп, телескоп):
\begin{enumerate}
    \item \textbf{Тонкая линза} рассматривается как:
      \begin{itemize}
       \item Луч 1: проходит через точку предмета и центр линзы (не преломляется).
       \item Луч 2: параллелен оптической оси, затем проходит через фокус.
      \end{itemize}
      Пересечение этих двух лучей даёт положение соответствующей точки изображения.
    \item \textbf{Микроскоп, телескоп} строятся последовательно: изображение, сформированное первой линзой, становится «предметом» для второй и т.д. При этом требуется несколько раз вызвать функцию преобразования.
\end{enumerate}

\section{Основные фрагменты кода}

Ниже для примера приводим несколько ключевых строк. Полный код см. в ноутбуке:

\begin{lstlisting}[language=Python, basicstyle=\small]
class Line:
    def __init__(self, x0, y0, z0, m, n, p):
        self.offset = np.array([x0, y0, z0], dtype=float)
        self.multiply = np.array([m, n, p], dtype=float)

    def get_dot(self, t):
        return self.offset + t * self.multiply

    def intersection(self, other):
        A = np.array([self.multiply, -other.multiply, 
                      np.cross(self.multiply, other.multiply)]).T
        B = other.offset - self.offset
        try:
            t, s, _ = np.linalg.solve(A, B)
        except np.linalg.LinAlgError:
            return None
        return self.get_dot(t)
\end{lstlisting}

\begin{lstlisting}[language=Python, basicstyle=\small]
def thin_lens_image_input_lines(
    input_image, f, a
):
    # вычисляем b = (a*f)/(a-f)
    # строим лучи: один через центр линзы, другой параллельно оси
    # ...
    # возвращаем выходное изображение
    return output_image, b, M, error
\end{lstlisting}

\section{Результаты}

В ноутбуке продемонстрировано несколько сценариев:

\subsection{1. Тонкая линза}

Исходное изображение (рис.~\ref{fig:lens_in}) проходит через лупу (тонкую линзу) при заданных $f$ и $a$. Результат --- увеличенное (или уменьшенное) изображение, которое может оказаться перевёрнутым, если $a>f$ (действительное изображение).

\subsection{2. Микроскоп}

Состоит из двух линз (объектив и окуляр), расположенных на некотором расстоянии. Пошагово получаем:
\begin{enumerate}
  \item Промежуточное изображение, даваемое объективом,
  \item Финальное увеличенное изображение после окуляра.
\end{enumerate}
В ноутбуке выведены результаты в виде двух-трёх картинок.

\subsection{3. Телескоп}

По аналогии: также две линзы, но настраиваем их так, чтобы параллельные лучи на входе давали параллельные лучи на выходе. Угловое увеличение $\gamma = f_{\text{об}} / f_{\text{ок}}$.

\section{Выводы}

\begin{enumerate}
  \item Реализован базовый инструментарий для определения \textbf{пересечения прямых и плоскостей} в 3D-пространстве.
  \item Данный инструментарий применён для \textbf{трассировки лучей} в простейших оптических моделях (тонкая линза, микроскоп, телескоп).
  \item Код позволяет ``проецировать'' цифровые изображения (набор точек) через модель оптической системы, получая новое (искажённое, увеличенное) изображение.
  \item Подобный подход может быть расширен для более сложных систем, многократных отражений/преломлений и реальных (толстых) линз.
\end{enumerate}

\begin{figure}[h!]
  \centering
  \includegraphics[width=0.32\textwidth]{example_in.png}
  \includegraphics[width=0.32\textwidth]{example_out1.png}
  \includegraphics[width=0.32\textwidth]{example_out2.png}
  \caption{Пример: исходное изображение (слева), промежуточное (по первой линзе) и финальное изображение (по второй линзе).}
  \label{fig:lens_in}
\end{figure}

\end{document}